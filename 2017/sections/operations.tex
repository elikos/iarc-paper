\section*{Operations} \label{sec:op}

\subsection*{Flight preparations} \label{subsec:op-prep}

There are many software and hardware pre- and post-flight checks in place. However, manual verifications are still required to ensure performance and safety during flights, since elements such as mechanical integrity cannot be automatically verified.

\subsubsection*{Pre-flight checklist}

The following is a checklist of elements and statuses that must be verified, at the very least, once before each extended flight session.

\vspace{-0.75cm}
\begin{enumerate} \itemsep -5pt
	\item Verify that the overall mechanical structure is undamaged and that the payload is securely mounted
	\item Verify that the propellers spin in the right directions and are properly tightened
	\item Verify that the Li-Po batteries are sufficiently charged ($ \approx 16.8 $~V) and well fastened
	\item Verify that the battery alarm is connected and functional
	\item Turn on and enable the kill switch transmitter
	\item Turn on the manual control transmitter
	\item Connect the batteries and power on the onboard computer, flight controller, and peripherals
	\item Verify that radio calibration and presets are correct
	\item Verify that the optical flow and computer vision cameras lenses are focused (and lens covers are removed)
	\item Calibrate inertial sensors (magnetometers, gyroscopes, accelerometers)
	\item Verify that telemetry and network communications are functional
\end{enumerate}

\subsubsection*{Post-flight checklist}

After each flight session or attempt, the following checks must be made.

\vspace{-0.75cm}
\begin{enumerate} \itemsep -5pt
	\item Properly terminate every process and data acquisition/logging
	\item Shut down the onboard computer and other peripherals
	\item Disconnect the batteries
	\item Turn off transmitters
	\item Verify that the battery levels are over the safety threshold and physically intact
	\item Verify that all component temperatures are within their normal operating temperatures
	\item Verify that every mount, propeller and screw is properly tightened (especially after a hard landing or a crash)
\end{enumerate}

\subsection*{Man/machine interface} \label{subsec:op-interface}

Prior to a flight, as mentioned in the \textit{Target Detection} section, a remote calibration tool for the cameras is used to adjust to the environment’s lighting.

In-flight, a SSH connection or a ROS remote connection allows us to monitor flight data over our Wi-Fi network. In case of emergency, the pilot’s transmitter offboard switch permits manual control at all time, and the kill switch’s dedicated transmitter cuts the power of all motors.
